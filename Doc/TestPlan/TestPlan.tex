\documentclass[12pt, titlepage]{article}

\usepackage{booktabs}
\usepackage{tabularx}
\usepackage{hyperref}
\hypersetup{
colorlinks,
citecolor=black,
filecolor=black,
linkcolor=red,
urlcolor=blue
}
\usepackage[round]{natbib}
\usepackage{graphicx}
\input{../Comments}

\begin{document}

\title{Stock Prediction System} 
\author{Renjie Zhang}
\date{\today}
\maketitle

\pagenumbering{roman}

\section{Revision History}

\begin{tabularx}{\textwidth}{p{3cm}p{2cm}X}
\toprule {\bf Date} & {\bf Version} & {\bf Notes}\\
\midrule
2017-10-06 & 1.0 & create\\
2017-10-09 & 1.1 & update\\
\bottomrule
\end{tabularx}

~\newpage

\section{Symbols, Abbreviations and Acronyms}

\renewcommand{\arraystretch}{1.2}
\begin{tabular}{l l} 
\toprule 
\textbf{symbol} & \textbf{description}\\
\midrule 
T & Test\\
SVM & Support Vector Machine\\
RDD &Resilient Distributed Datasets\\
K & Kernel function\\
X & features of the data (date , price)\\
C& The price from different days\\
$\beta$ & The error modifier\\
y& The result between 1 and -1\\ 
\bottomrule
\end{tabular}\\



\newpage

\tableofcontents

\listoftables

\listoffigures

\newpage

\pagenumbering{arabic}



\section{General Information}
This Document is a test plan of the software-Stock Prediction System which is a tool used to predict the stock prices using machine learning. 

\subsection{Purpose}
This test plan describes the testing methods that will drive the testing of the Stock Prediction System and give a guide to the users about the QA. This document includes the descriptions of testing tool, testing functions, unit testing method and system test. Based on these testing, the users may cache the functional and non functional issues from the first release of the software and find out the improvement. 
\subsection{Scope}
The testing plan will cover both system testing and unit testing. The functions are the data input, data format validation, calculation of SVM, data plot, Spark RDD Distributed System, and the output of the result. Basically every part of the system will be tested.In this plan, most of the testing will be done by unit testing.
\subsection{Overview of Document}
This document explains the testing plan for the software - Stock Prediction System. It includes the briefcase of software description, the different testing methods - unit testing and integration testing, testing tools. It will cover both functional and non functional requirements. 
\section{Plan}
In this section, the general description of the testing plan for the software and corresponding testing tools will be introduced.
\subsection{Software Description}
The Stock Prediction System is used to analyze the future trend of stocks. The prediction was provided by machine learning algorithms based on the historical data. 
The system will be run on a big data platform (Spark), in order to obtain the more accurate results. In this case, we need to setup a distributed system to support Spark.
\subsection{Test Team}

Renjie Zhang

\subsection{Automated Testing Approach}
NA
\subsection{Verification Tools}
Python unit testing : This framework was included in Python standard library. It is easy to use by people familiar with the xUnit frameworks, strong support for test organization and reuse via test suites\\
Wing : A popular Python IDE which contents the code checking and indent correction.\\
PyChecker (Optional) : A source checking tools which finds problems that are typically caught by a compiler for less dynamic languages; imports each module before checking it.\\



% \subsection{Testing Schedule}
% See Gantt Chart at the following url ...

\subsection{Non-Testing Based Verification}
NA

\section{System Test Description}
This section describes the testing on the system environment, such as the data file input and spark distribution system. Users need to make sure the application run on the system without any issues. 
\subsection{Tests for Functional Requirements}
The functional requirements includes the data input, the data format verification, the Spark RDD transaction, the plot generation and the display of the output. The software must fulfill all of the functional requirements without issues.

\subsubsection{Data Input}
The system needs to load a dataset from a CSV file. This test ensures the data input method. 
\paragraph{File loading Testing }

\begin{enumerate}

\item{File is loaded successfully\\}

Type: Functional, Manual, Static etc.
Initial State: NA
Input: File Path
Output: Successful Message
How test will be performed: System tries to load the data set file based on the file name and location without issues.
\item{Input Data Validation\\}

Type: Functional, Manual
Initial State: NA
Input: Data from the file
Output: Successful message
How test will be performed: System have to ensure the data type and format is correct for each column of the file such as the pattern of the date, the formats of the price and number of digits of decimals. Shown in Figure 1

~\newline
\begin{figure}[h!]
\begin{center}
%\rotatebox{-90}
{
\includegraphics[width=0.5\textwidth]{amazon.png}
}
\caption{\label{Input Data}}
\end{center}
\end{figure}

\end{enumerate}

\subsubsection{Distributed System}
This section focus on the distributed system of Spark platform. It guides the users to test the data flow on the spark system. RDD is the format of data flow in Spark. testers need to double check the data was distributed into each workers through RDD.
\paragraph{ Spark RDD Testing}


\begin{enumerate}

\item{Data is distributed to workers\\}

Type: Functional, Manual, Static etc.
Initial State: NA
Input: An Array from the driver
Output: An updated array with marks of each worker
How test will be performed: There will be an input array to the driver, driver assign the array to each workers. Workers add a mark with its identification to elements of the array and return it back to driver. Shown in Figure 2.

~\newline
\begin{figure}[h!]
\begin{center}
%\rotatebox{-90}
{
\includegraphics[width=0.5\textwidth]{sparkrdd.png}
}
\caption{\label{Input Data}}
\end{center}
\end{figure}

\end{enumerate}

\subsubsection{Testing on The Algorithm}
This section gives the guide to the testing on the Support Vector machine algorithms. It covers the plot test and the calculation of functions in SVM.
\paragraph{SVM Testing}

\begin{enumerate}

\item{Plot Testing\\}

Type: Functional, Manual, Static etc.
Initial State: NA
Input: A pre-defined Array
Output: A correct plot
How test will be performed: System reads the data and generate a plot using the price and the date as the x and y axis.

\item{SVM Calculation \\}

Type: Functional, Manual, Static etc.
Initial State: NA
Input: A collection of data
Output: correct results
How test will be performed: The detail of the functions related to SVM Calculation can be found in Unit Test

\end{enumerate}

\subsection{Tests for Non functional Requirements}
Non functional requirements are not as prior as functional requirements, but it is still necessary to have tests on them in order to improve the software performance and quality.
\subsubsection{Big Data System Testing}
In this section, the only requirement is the performance.
\paragraph{ Performance Testing}

\begin{enumerate}

\item{Performance of the workers\\}

Type: Manual
Initial State: NA
Input/Condition: NA
Output/Result: Result of the time cost\\
How test will be performed: It is a good way to increase the performance by increase the number of works. Try to increase one more worker and compare the time cost on data training and testing.


\end{enumerate}

\newpage
\subsection{Traceability Between Test Cases and Requirements}

\begin{table}[h!]
\centering
\begin{tabular}{|c|c|c|c|c|c|c|c|}
\hline
& Input Data & Plot & Data Validation & Calculation & Output &Performance\\
\hline
Input Data testing &X & & & & & \\ \hline
Data Validation &X & &X & & & &X \\ \hline
RDD Testing &X & & & & & & \\ \hline
Plot testing & &X & & & &X \\ \hline
Kernelling & & & & X& & &X \\ \hline
Volatility Testing & & & &X & & & \\ \hline 
Momentum Testing & & & &X & & & \\ \hline 
Prediction Testing & & & &X &X & & \\ \hline 
Performance Testing & & & & & & X& \\ \hline 

\end{tabular}
\caption{Traceability Matrix Showing the Connections Between Requirements and Test Cases}
\label{Table:R_trace}
\end{table}

% \section{Tests for Proof of Concept}

% \subsection{Area of Testing1}
% \paragraph{Title for Test}

% \begin{enumerate}

% \item{test-id1\\}

% Type: Functional, Dynamic, Manual, Static etc.
% Initial State: 
% Input: 
% Output: 
% How test will be performed: 
% \item{test-id2\\}

% Type: Functional, Dynamic, Manual, Static etc.
% Initial State: 
% Input: 
% Output: 
% How test will be performed: 

% \end{enumerate}

% \subsection{Area of Testing2}

% ...
\section{Unit Testing Plan}
The unit testing plan will involve the following modules: Load Data, SVM Kernelling, Volatility Calculating, Momentum Calculating, Predict and Output.\\
\begin{enumerate}

\item{ Load Data\\}
The detail of the data loading was explained on 5.1.1
\item{ Kernelling\\}
In machine learning, kernel methods are a class of algorithms for pattern analysis, whose best known member is the support vector machine (SVM)
RBF Kernel is used for price forecasting. A correct result is expected by inputting the parameters to the equation \\
$K\left (X_i,X_k\right )=\exp \left ( -\frac1{\delta^2}\sum_{n}^{j=1}(X_{ij}-X_{kj})^2 \right )$ \\

\item{ Calculate Volatility\\}
This function is use to calculate the price volatility and index volatility. Since the calculation is similar they share the same function. A correct result is expected by inputting the parameters to the equation \\
$\frac{\sum_{i=t-n+1}^{t} \frac{C_i-C_{i-1}}{C{i-1}}}{n}$ \\ 

\item{ Calculate Momentum\\}
This function is use to calculate the price momentum and index momentum. Since the calculation is similar they share the same function. A correct result is expected by inputting the parameters to the equation \\
$\frac{\sum_{i=t-n+1}^{t} \frac{C_i-C_{i-1}}{C{i-1}}}{n}$ \\ 

\item{Predict \\}
The Predict module receives a set of parameters calculated from the previous functions and returns a result. A correct result is expected by inputing the parameters to the equation \\\\
$y=\beta _0+\sum {a_iy_iK(x(i),x)}$\\

\end{enumerate} 


\bibliographystyle{plainnat}

\bibliography{SRS}

\newpage

\section{Appendix}

NA

\subsection{Symbolic Parameters}

NA

\subsection{Usability Survey Questions?}

NA

\end{document}

