\documentclass[12pt, titlepage]{article}

\usepackage{amsmath, mathtools}

\usepackage[round]{natbib}
\usepackage{amsfonts}
\usepackage{amssymb}
\usepackage{graphicx}
\usepackage{colortbl}
\usepackage{xr}
\usepackage{hyperref}
\usepackage{longtable}
\usepackage{xfrac}
\usepackage{tabularx}
\usepackage{float}
\usepackage{siunitx}
\usepackage{booktabs}
\usepackage{multirow}
\usepackage[section]{placeins}
\usepackage{caption}
\usepackage{fullpage}
\hypersetup{
bookmarks=true, % show bookmarks bar?
colorlinks=true, % false: boxed links; true: colored links
linkcolor=red, % color of internal links (change box color with linkbordercolor)
citecolor=blue, % color of links to bibliography
filecolor=magenta, % color of file links
urlcolor=cyan % color of external links
}

\usepackage{array}

%% Comments

\usepackage{color}

\newif\ifcomments\commentstrue

\ifcomments
\newcommand{\authornote}[3]{\textcolor{#1}{[#3 ---#2]}}
\newcommand{\todo}[1]{\textcolor{red}{[TODO: #1]}}
\else
\newcommand{\authornote}[3]{}
\newcommand{\todo}[1]{}
\fi

\newcommand{\wss}[1]{\authornote{blue}{SS}{#1}}
\newcommand{\an}[1]{\authornote{magenta}{Author}{#1}}

\newcounter{mnum}
\newcommand{\mthemnum}{M\themnum}
\newcommand{\mref}[1]{M\ref{#1}}

\newcommand{\progname}{Stock Prediction System}

\begin{document}

\title{Module Interface Specification for Stock Prediction System}

\author{Renjie Zhang}

\date{\today}

\maketitle

\pagenumbering{roman}

\section{Revision History}

\begin{tabularx}{\textwidth}{p{3cm}p{2cm}X}
\toprule {\bf Date} & {\bf Version} & {\bf Notes}\\
\midrule
22/11/2017 & 1.0 & Create\\
28/11/2017 & 1.1 & Update\\
16/12/2017 & 1.2 & Update\\
\bottomrule
\end{tabularx}

~\newpage

\section{Symbols, Abbreviations and Acronyms}

NA

\newpage

\tableofcontents

\newpage

\pagenumbering{arabic}

\section{Introduction}

The following document details the Module Interface Specifications for
Stock Prediction System which is used to predict the future stock price based on the historical data. Complementary documents include the System Requirement Specifications
and Module Guide. The full documentation and implementation can be
found at \url{https://github.com/renjiezhang/CAS-741}.\\

\section{Notation}

The structure of the MIS for modules comes from \citet{HoffmanAndStrooper1995},
with the addition that template modules have been adapted from
\cite{GhezziEtAl2003}. The mathematical notation comes from Chapter 3 of
\citet{HoffmanAndStrooper1995}. For instance, the symbol := is used for a
multiple assignment statement and conditional rules follow the form $(c_1
\Rightarrow r_1 | c_2 \Rightarrow r_2 | ... | c_n \Rightarrow r_n )$.

The following table summarizes the primitive data types used by \progname. 

\begin{center}
\renewcommand{\arraystretch}{1.2}
\noindent 
\begin{tabular}{l l p{7.5cm}} 
\toprule 
\textbf{Data Type} & \textbf{Notation} & \textbf{Description}\\ 
\midrule
character & char & a single symbol or digit\\
integer & $\mathbb{Z}$ & a number without a fractional component in (-$\infty$, $\infty$) \\
natural number & $\mathbb{N}$ & a number without a fractional component in [1, $\infty$) \\
real & $\mathbb{R}$ & any number in (-$\infty$, $\infty$)\\
List &list & a list of objects\\
Stock Record & record & a Record has two elements a date by string and a price by real number\\
Date &date & a date type with format yyyy-mm-dd\\
\bottomrule
\end{tabular} 
\end{center}

\noindent
The specification of \progname \ uses some derived data types: sequences, strings, and
tuples. Sequences are lists filled with elements of the same data type. Strings
are sequences of characters. Tuples contain a list of values, potentially of
different types. In addition, \progname \ uses functions, which
are defined by the data types of their inputs and outputs. Local functions are
described by giving their type signature followed by their specification.

\section{Module Decomposition}

The following table is taken directly from the Module Guide document for this project.

\begin{description}
\item [\refstepcounter{mnum} \mthemnum \label{mHH}:] Hardware-Hiding Module
\item [\refstepcounter{mnum} \mthemnum \label{mMain}:]Main Module
\item [\refstepcounter{mnum} \mthemnum \label{mInput}:] Data Input Module
\item [\refstepcounter{mnum} \mthemnum \label{mVolatility}:] Price Volatility Module
\item [\refstepcounter{mnum} \mthemnum \label{mMomentum}:] Price Momentum Module
\item [\refstepcounter{mnum} \mthemnum \label{mPrediction}:]Prediction Module
\item [\refstepcounter{mnum} \mthemnum \label{mKernelling}:]Kernelling Module
\item [\refstepcounter{mnum} \mthemnum \label{mPlot}:] Plot Module
\item [\refstepcounter{mnum} \mthemnum \label{mRDD}:]  RDD Module
\end{description}

\begin{table}[h!]
\centering
\begin{tabular}{p{0.3\textwidth} p{0.6\textwidth}}
\toprule
\textbf{Level 1} & \textbf{Level 2} \\
\midrule

{Hardware-Hiding} & ~ \\
\midrule


\multirow{7}{0.3\textwidth}{Behaviour-Hiding Module}
& Main Module\\
& Data Input Module\\
& Price Volatility Module\\
& Price Momentum Module\\
& Prediction Module\\
& Data Plot Module\\
\midrule

\multirow{1}{0.3\textwidth}{Software Decision Module}& RDD Module\\
&Kernelling Module\\

\bottomrule

\end{tabular}
\caption{Module Hierarchy}
\label{TblMH}
\end{table}

\newpage
~\newpage

\section{MIS of Main Module (\mref{mMain}) } 

\subsection{Module}
Main
\subsection{Uses}
Hardware-Hiding Module (\mref{mHH})

\subsection{Syntax}
NA
\subsubsection{Exported Access Programs}

\begin{center}
\begin{tabular}{p{2cm} p{4cm} p{4cm} p{2cm}}
\hline
\textbf{Name} & \textbf{In} & \textbf{Out} & \textbf{Exceptions} \\
\hline

main & - &- & - \\
\hline
\end{tabular}
\end{center}

\subsection{Semantics}
\subsubsection{State Variables}
NA
\subsubsection{Enviroment Variables}
ndxtPrices: list of real numbers
ndxtDates: list of dates
numDaysArray : list of integers
numDayAheadArray : list of integers
\subsubsection{Access Routine Semantics}

\noindent main():
\begin{itemize}
\item transition: Calls the Data Input Modules, Volatility Module, Momentum Module and Predict Module with parameters and retrieve the returned data from them. \\
-DataInput(filePath)\\
-ndxtVolatilityArray = sc.parallelize(GetPriceVolatility(daysAhead,numDayStock, ndxtPrices)).collect()\\ 
-ndxtMomentumArray = sc.parallelize(GetMomentum(daysAhead,numDayStock, ndxtPrices)).collect()\\
-Predict(company,daysAhead,numDayStock,ndxtVolatilityArray,ndxtMomentumArray)
\item output: NA
\item exception: NA
\end{itemize}

~\newpage

\section{MIS of Data Input Module (\mref{mInput}) } 

\subsection{Module}
Input Module
\subsection{Uses}
Main Module (\mref{mMain})
\subsection{Syntax}

\subsubsection{Exported Access Programs}

\begin{center}
\begin{tabular}{p{2cm} p{4cm} p{4cm} p{2cm}}
\hline
\textbf{Name} & \textbf{In} & \textbf{Out} & \textbf{Exceptions} \\
\hline

ReadCSV & string & record list & IOError \\
\hline
\end{tabular}
\end{center}


\subsection{Semantics}
\subsubsection{State Variables}
NA
\subsubsection{Environment Variables}
dataSet : record list
\subsubsection{Access Routine Semantics}

\noindent ReadCSV():
\begin{itemize}
\item transition: NA
\item output: A record list of the date and price 
\item exception: \\
-IOError: Invalid file name and path. Invalid column name and data format
in the file. 
\end{itemize}

~\newpage


\section{MIS of Price Volatility Module (\mref{mVolatility}) } 

\subsection{Module}
Volatility Module
\subsection{Uses}
Data Input Module (\mref{mInput})
~\newline
Main Module (\mref{mMain})
~\newline
 RDD Module (\mref{mRDD})
\subsection{Syntax}

\subsubsection{Exported Access Programs}

\begin{center}
\begin{tabular}{p{4cm} p{2cm} p{2cm} p{4cm}}
\hline
\textbf{Name} & \textbf{In} & \textbf{Out} & \textbf{Exceptions} \\
\hline

GetPriceVolatility & $\mathbb{R}^n$ ,$\mathbb{N}^n$ & $\mathbb{R}^n$ & NotFittedError \\
\hline
\end{tabular}
\end{center}

\subsection{Semantics}
\subsubsection{State Variables}
NA

\subsubsection{Access Routine Semantics}

\noindent GetPriceVolatility():
\begin{itemize}
\item transition: NA
\item output: \\
volatilityArray: A list of real numbers represents the price volatility
\item exception:
~\newline
NotFittedError : Improper price such as negative number.
\end{itemize}

~\newpage

\section{MIS of Price Momentum Module (\mref{mMomentum}) } 

\subsection{Module}
Momentum Module
\subsection{Uses}
Data Input Module (\mref{mInput})
~\newline
Main Module (\mref{mMain})
~\newline
RDD Module (\mref{mRDD})
\subsection{Syntax}


\subsubsection{Exported Access Programs}

\begin{center}
\begin{tabular}{p{4cm} p{2cm} p{2cm} p{4cm}}
\hline
\textbf{Name} & \textbf{In} & \textbf{Out} & \textbf{Exceptions} \\
\hline

GetPriceMomentum & $\mathbb{R}^n$ ,$\mathbb{N}^n$ & $\mathbb{R}^n$ & NotFittedError \\
\hline
\end{tabular}
\end{center}

\subsection{Semantics}
\subsubsection{State Variables}
NA

\subsubsection{Access Routine Semantics}

\noindent GetPriceMomentum():
\begin{itemize}
\item transition: NA
\item output:\\
momentumArray : A list of real number for the price momentum
\item exception:
~\newline
NotFittedError : Improper price such as negative number.
\end{itemize}

~\newpage
\section{MIS of Prediction Module (\mref{mPrediction}) } 

\subsection{Module}
Predict Module
\subsection{Uses}
Data Input Module (\mref{mInput})
~\newline
Main Module (\mref{mMain})
~\newline
Volatility Module(\mref{mVolatility})
~\newline
Momentum Module(\mref{mMomentum})
~\newline
Kernelling Module(\mref{mKernelling})
~\newline
RDD Module(\mref{mRDD})

\subsection{Syntax}

\subsubsection{Exported Access Programs}

\begin{center}
\begin{tabular}{p{2cm} p{4cm} p{2cm} p{2cm}}
\hline
\textbf{Name} & \textbf{In} & \textbf{Out} & \textbf{Exceptions} \\
\hline

Predict & string, $\mathbb{R}^n$ ,$\mathbb{N}^n$ & $\mathbb{R}$ &- \\
\hline
\end{tabular}
\end{center}

\subsection{Semantics}
\subsubsection{State Variables}
NA
\subsubsection{Environment Variables}
Predict()

~\newline
volatilityArray: A list of real numbers for the price volatility list calculated from the Price Volatility Model\\
~\newline
momentumArray: A list of real numbers for the price momentum list calculated from the Price Momentum Model\\
~\newline
featureX: A list of array which consists four real number elements : price volatility, price momentum, index volatility and index momentum\\ 
~\newline
featureY: A list of integers (1 or -1)\\ 
~\newline
Kernel mode: kernel='rbf
\subsubsection{Access Routine Semantics}

\noindent Predict():
\begin{itemize}
\item transition: NA
\item output:\\
score :A real number for the percentage of the possibility
\item exception: NA
\end{itemize}

~\newpage

\section{MIS of Plot Module (\mref{mPlot}) } 

\subsection{Module}
Plot Module
\subsection{Uses}
Data Input Module (\mref{mInput})
~\newline
Main Module (\mref{mMain})
\subsection{Syntax}

\subsubsection{Exported Access Programs}

\begin{center}
\begin{tabular}{p{2cm} p{4cm} p{2cm} p{2cm}}
\hline
\textbf{Name} & \textbf{In} & \textbf{Out} & \textbf{Exceptions} \\
\hline

Plot & char, $\mathbb{R}^n$ &- &- \\

\hline
\end{tabular}
\end{center}

\subsection{Semantics}
\subsubsection{State Variables}
Dates: The array of the dates of each record\\ 
~\newline
Prices: The array of the prices of each record

\subsection{Semantics}
\subsubsection{Enviroment Variables}
Dates: The array of the dates of each record\\ 
~\newline
Prices: The array of the real number for the prices of each record\\ 
\subsubsection{Access Routine Semantics}

\noindent Plot():
\begin{itemize}
\item transition: NA
\item output: NA
\item exception: NA
\end{itemize}

%%%%%%%%%%%%%%%%%%%%%%%%%%%
\newpage

\subsection{Reference}
\bibliographystyle {plainnat}
\bibliography {../../../ReferenceMaterial/References}

Modeling high-frequency limit order book dynamics with support vector machines PDF 2013
~\newline
Predicting Stock Price Direction using Support Vector Machines PDF 2015


\end{document}

